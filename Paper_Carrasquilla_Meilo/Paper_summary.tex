\documentclass[a4paper,11pt]{jsarticle}


% 数式
\usepackage{amsmath,amsfonts}
\usepackage{bm}
% 画像
\usepackage[dvipdfmx]{graphicx}


\begin{document}

\title{Machine learning phases of matter}
\author{須賀勇貴}
\date{最終更新日:\today}
\maketitle
ニューラルネットワークは,教師あり機械学習により,相や凝縮系の相転移を識別するために使用することができる.現代のソフトウェアライブラリを介して簡単にプログラム可能であることを示し,標準の順伝播ニューラルネットワークは,モンテカルロでサンプリングされた生の状態構成から直接複数の種類の秩序パラメータを検出するために訓練できることを示す.さらに,クーロン相のような異常に非自明な状態や、畳み込みニューラルネットワークに変更された場合には従来の秩序パラメータのないトポロジカル相も検出することができる.これらの分類が,ハミルトニアンや一般的な相互作用の局所性の知識なしにニューラルネットワーク内で発生することを示している.これらの結果は,凝縮系と統計物理学の分野で基礎研究ツールとしての機械学習の力を示している.\par

\newpage
凝縮物質物理学は,電子,原子核,磁気モーメント,原子,またはキュビットなど,非常に複雑なアンサンブルを研究する学問である [1].この複雑性は,古典的または量子的な状態空間のサイズに反映され,そのサイズは粒子の数に応じて指数的に増加する.この指数的な成長は,機械学習で一般的に遭遇する「次元の呪い」を思い起こさせます.つまり,学習する対象の関数には,寸法(たとえば画像の特徴の数)に指数的に増加する訓練データが必要になる.この呪いにもかかわらず,機械学習コミュニティは,複雑なデータセットを認識,分類,および特徴付けするための驚くべき能力を持ついくつかの技術を開発している.この成功を踏まえて,凝縮物質物理学の領域で,特に微視的なハミルトニアンに強い相互作用が含まれ,相と相転移の研究に通常数値シミュレーションが使用される場合に,そのような技術が適用できるかどうかは自然な疑問である [2, 3].私たちは,全結合および畳み込みニューラルネットワークなどの現代の機械学習アーキテクチャが,凝縮物質物理学のさまざまなシステムで相と相転移を識別するための補完的なアプローチを提供できることを示している [4].モンテカルロサンプリングによって得られたデータセットでニューラルネットワークを訓練することは,物理モデルの相と相境界の教師あり学習のための特に強力で簡単なフレームワークを提供し,Theano [5]やTensorFlow [6]などの手に入りやすいツールから簡単に構築できる.\par
従来,凝縮物質系のフェーズの研究は,さまざまな状態の基本的な物理的構造を明らかにするために注意深く設計されたツールを使用して行われている.その中でも最も強力なものの一つがモンテカルロシミュレーションであり,これは次の2つのステップから構成されます:状態空間に対する確率的な重要性サンプリングと,これらのサンプルから計算された物理量の推定子の評価 [3].これらの推定子は,さまざまな物理的な刺激に基づいて構築される.たとえば,特定の熱容量のような類似の実験的測定が容易に利用できる場合や、,秩序パラメータのようなより抽象的な理論的なデバイスのエンコードなどである [1].しかし,物質のユニークで技術的に重要な状態は,標準の推定子では簡単には同定できないかもしれない.実際,トポロジカルに整列した状態などの一部の非常に求められる相については,確実な同定にはエンタングルメントエントロピーなどの非常に高価な手法(かつ実験的に挑戦的な手法[8])が必要かもしれない [9, 10].\par





\end{document}