\documentclass[a4paper,11pt]{jsarticle}


% 数式
\usepackage{amsmath,amsfonts}
\usepackage{bm}
% 画像
\usepackage[dvipdfmx]{graphicx}


\begin{document}

\title{相転移とは何か}
\author{須賀勇貴}
\date{最終更新日:\today}
\maketitle
高橋・西森先生の「相転移・臨界現象とくりこみ群」の2章を参考にした.
\vspace{0.5cm}

\begin{itemize}
  \item 相転移とは何か?
  \item 相転移に伴い生じる対称性の自発的破れ
\end{itemize}

\section{統計力学と熱力学関数}
統計力学の基本の復習として,熱力学量の定義とそれらの関係についてと熱力学極限の重要性について説明する.

\subsection{カノニカル分布}
ここで扱う対象は微視的な要素が多数集まってできる巨視的な系であり,熱力学のより記述される.熱力学は,熱エネルギーの移動によって引き起こされるさまざまな現象を扱うための体系である.エネルギーやエントロピーなどの熱力学量の関係を規定するが,対象とする系の微視的な詳細を問わない.それはつまり,熱力学は与えられた系(ハミルトニアン)に対して熱力学量を具体的に計算することができないことを意味する.そこに統計力学の存在意義が生じる.\par
統計力学は,熱力学において扱われる熱力学量を微視的な確率分布模型から具体的に計算する処方箋を与える.等重率の原理(仮説)を出発点にして構成された理論であり,熱力学との統合性が構成の指針となる.統計力学を用いて計算された熱力学量は,たいていの場合,熱力学が要請する性質を備えている.しかしながら,統計力学の基本的前提条件に沿わない模型に統計力学の枠組に適用すると,熱力学と矛盾する結果が得られることがある.例えば,負の温度という概念がある.これはすべての許されるエネルギー値の単純平均より指定値$E$が大きくなるとき現れるものである.熱力学において定義される温度は非負の量であり,負の温度に対応する状態の熱力学的な意味は不明である.また,相互作用が非常に長い距離に及んでいる系に統計力学を適用しようとすると,負の比熱のような異常なふるまいが得られることがある.長距離相互作用のある系の統計力学は未解決な部分が多く,今後の課題として残されている.\par
相転移は熱力学量を表す熱力学関数の特異性として表現されるため,統計力学を適用する際にはその適用限界に注意しなければならない.特異性が物理的に意味のあるものなのかどうかを見極める必要がある.そのため,統計力学を用いて評価される熱力学関数にどのような特異性が許されるかを理解することがこの章での主要な目的である.\par
統計力学で最も多く用いられる確率分布模型は,カノニカル分布(canonical distribution)である.考察対象である系の大きさと比べて十分大きい熱浴を用いて,系と熱浴の間にエネルギーのやり取りを許す.このとき,系の状態を指定する指標となるのはエネルギーではなく,熱浴の状態数を用いて定義される温度である.温度一定であることが,系が熱浴と接して熱平衡状態を保っていることを表している.このような設定の下で系が各々の微視的状態を取る確率が与えられ,その確率分布を用いて熱力学量が求められる.\par
カノニカル分布に基づいて熱力学量を計算する処方箋は次のとおりである.多体系のハミルトニアン$H$が与えられたとき,分配関数を
\begin{equation}
  Z = \operatorname{Tr}e^{-\beta H}
\end{equation}
と定義する.$\beta=1/T$は逆温度と呼ばれる.本来はBoltzmann定数$k_B$を用いて$\beta=\frac{1}{k_B T}$とするが,ここでは$k_B=1$とする自然単位系を用いるため省略する.このとき温度$T$はエネルギーの単位をもつ.$\operatorname{Tr}$はとりうる微視的状態すべてについての和を表している.和が離散的か連続的かは考えている系による.分配関数自体は熱力学量ではないが,ここ型すべての熱力学量が計算される.\par
Helmholtzの自由エネルギーは次のように定義される.
\begin{equation}
  F = \frac{1}{\beta}\ln{Z}
\end{equation}
この関数が熱力学によって規定される自由エネルギーに対応しており,以下の議論で中心的な役割を果たす.自由エネルギーが決まれば,熱力学の関係式を用いて他の熱力学量も計算することができる.内部エネルギー,または単にエネルギーは
\begin{equation}
  E = \frac{\partial}{\partial \beta}(\beta F)
\end{equation}
であり,エントロピーは
\begin{equation}
  S = -\frac{\partial F}{\partial T}
\end{equation}
と表される.これらの関数は
\begin{equation}
  F = E -TS
\end{equation}
によって特徴づけられる.実際,
\begin{align*}
  E - TS
   & = \frac{\partial}{\partial \beta}(\beta F) - \beta \frac{\partial F}{\partial \beta}                                  \\
   & = -\frac{\partial}{\partial \beta} \ln{Z} + \beta \frac{\partial}{\partial \beta}\left( \frac{1}{\beta}\ln{Z} \right) \\
   & = -\frac{\partial}{\partial \beta} \ln{Z} - \frac{1}{\beta} \ln{Z} + \frac{\partial}{\partial \beta} \ln{Z}           \\
   & = - \frac{1}{\beta} \ln{Z} = F
\end{align*}
となって,正しいことが確かめられる.そして,この$F$と$E$の関係はLegendre変換にほかならない.自由エネルギーは温度の関数であるが,エネルギーはエントロピーの関数である.また,Helmholtzの自由エネルギーは体積$V$の関数でもあるのだが,Legendre変換を用いるとGibbsの自由エネルギー
\begin{equation}
  G = F + pV, \ \ p = -\frac{\partial F}{\partial V}
\end{equation}
を定義することができる.これは温度と圧力の関係である.\par
以上の処方箋によって,微視的なハミルトニアンと巨視的な熱力学量が直接結びつけられる.このことをもう少し詳しく見てみよう.ハミルトニアンの固有値が$E_n$で表されるとする.$n$はとりうる固有状態を表す指標であり,個々の微視的状態を表す.分配関数は次のように書ける.
\begin{equation}
  Z = \sum_{n} e^{-\beta E_n}
\end{equation}
これを上の公式に代入すると,エネルギーの表式が得られる.
\begin{align}
  E
    & = - \frac{\partial}{\partial \beta} \ln{Z} \notag                                            \\
    & = - \frac{1}{Z}\frac{\partial Z}{\partial \beta} \notag                                      \\
    & = - \frac{1}{Z}\frac{\partial}{\partial \beta} \left( \sum_{n} e^{-\beta E_n} \right) \notag \\
    & = \frac{1}{Z} \sum_{n} E_n e^{-\beta E_n}                                                    \\
  ( & = \sum_{n} E_n P_n) \notag                                                                   \\
\end{align}
つまり,状態$n$が実現される確率はBoltzmann因子$\exp{-\beta E_n}$を分配関数で割ったもの
\begin{equation}
  P_n = \frac{1}{Z} e^{-\beta E_n}
\end{equation}
で表される.また,エントロピーは
\begin{equation}
  S = -\sum_{n} P_n \ln{P_n}
\end{equation}
と書ける.$0 \leq P_n \leq 1$であるから,エントロピーは非負の量であるという熱力学的な要請を満たしている.\par
これらの式を用いて,系の状態や熱力学量のとりうる値は定性的に理解することができる.絶対零度極限$\beta \rightarrow \infty$では,系の基底状態,つまり$E_n$が最も小さい状態のみをとることが確率分布の表式(\ref{})からわかる.基底状態$n=0$を縮退していないとすると,$P_n=\delta_{n0}$となる(熱力学第3法則).また,$F=E-TS$より,絶対零度では自由エネルギーはエネルギーに等しい.逆の高温極限$\beta \rightarrow 0$では,各状態の実現確率$P_n$はすべての状態で等しくなる.状態総数を$M$とすると,$P_n = 1/M$である.このとき,エネルギーはすべての状態のエネルギーの平均$E=\frac{1}{M}\sum_{n=1}^{M}E_n$,エントロピーは状態数の対数$S=\ln{M}$で与えられる.有限温度では低温極限と高温極限の結果を内挿するふるまいが得られるはずである.エネルギーが大きい状態ほど$P_n$は小さいが,状態の数は増えていくため,分配関数の計算は非自明なものとなる.\par
自由エネルギーやエントロピーを直接測定することは難しいため,それらの変化率を表す微分量が重要になる.例えば,エネルギーまたはエントロピーを温度で微分した量として,熱容量
\begin{equation}
  C = \frac{\partial E}{\partial T} \equiv T \frac{\partial S}{\partial T}
\end{equation}
が定義される.熱容量の統計力学的な表現は次の通りである.
\begin{equation}
  C = \beta^2 \left[ \sum_n P_n E_n^2 - \left( \sum_n P_n E_n \right)^2 \right]
\end{equation}
つまり,カノニカル分布におけるエネルギーのゆらぎが熱容量を与える.この量は明らかに非負であり,温度を上げるとエネルギーおよびエントロピーが上昇するという熱力学的な要請を満たしている.


\end{document}