\documentclass[a4paper,11pt]{jsarticle}


% 数式
\usepackage{amsmath,amsfonts}
\usepackage{bm}
% 画像
\usepackage[dvipdfmx]{graphicx}


\begin{document}

\title{イジング模型のくりこみ群}
\author{須賀勇貴}
\date{最終更新日:\today}
\maketitle
高橋・西森先生の「相転移・臨海現象とくりこみ群」を参考にまとめた.
\vspace{0.5cm}

\section{1次元イジング模型}
以下では,1次元イジング模型の解析を具体的に行う.
\subsection{粗視化}
まずは,1次元イジングモデルの場合における粗視化を考えていく.各格子点に置かれたスピンの系を粗視化するということは,となり合ったいくつかのスピンをまとめて一つに見るということを意味する.考えられる中で最も単純は粗視化として,一つおきにスピンについて和をとるという方法をとる.\par
1次元イジング模型の分配関数は,$K=\beta J, \ H = \beta h$として
\begin{equation}
  Z = \sum_{\sigma_1,\sigma_2,\dots,\sigma_N} \exp{\left( K\sum_{i=1}^{N} \sigma_{i}\sigma_{i+1}
    + H \sum_{i=1}^{N} \sigma_i \right)}
\end{equation}
と書ける.スピンの数を$N$とし,この値は偶数であると仮定する.また,周期的境界条件$\sigma_{N+1}=\sigma_{1}$を課す.ここで,偶数番目のスピンについて和をとることを考えよう.例として,スピン$\sigma_{2}$に関する部分和を抜き出すと,
\begin{equation}
  \sum_{\sigma_{2} = \pm 1} \exp{(K\sigma_1\sigma_2 + H\sigma_2 + K\sigma_2\sigma_3)}
  = e^{K\sigma_1 + H + K\sigma_3} + e^{-K\sigma_1 - H - K\sigma_3}
\end{equation}
となる.右辺はスピン変数$S_1$と$S_3$を含む関数であるが,スピン変数は2乗すれば1になるため,定数および,$S_1, \ S_3. \ S_1 S_3$の線形結合で表すことができる.また,指数関数の形に書くこともでき,上式は
\begin{equation}
  \exp{\left( A' + \frac{\delta H}{2}S_1 + K'S_1S_3 + \frac{\delta H}{2}S_3 \right)}
\end{equation}
と元のハミルトニアンと似た形で表すことができる.また,$S_2$を消去することによって,もともと存在していなかった$S_1$と$S_3$の相互作用項が生じていることが分かる.\par
以上より,1次元イジング模型において部分和をとることで,
\begin{align*}
  Z
   & = \sum_{\sigma_1,\sigma_2,\dots,\sigma_N} \exp{\left( K\sum_{i=1}^{N} \sigma_{i}\sigma_{i+1}
  + H \sum_{i=1}^{N} \sigma_i \right)}                                                                                                                     \\
   & = \sum_{\{\sigma_{\text{odd}}\}} \exp{\left( H\sum_{i=1}^{N/2} \sigma_{2i-1} \right)}
  \sum_{\{\sigma_{\text{even}}\}} \exp{\left( K\sum_{i=1}^{N} \sigma_i \sigma_{i+1} + H \sum_{i=1}^{N/2} \sigma_{2i} \right)}                              \\
   & = \sum_{\{\sigma_{\text{odd}}\}} \exp{\left( H\sum_{i=1}^{N/2} \sigma_{2i-1} \right)}
  \sum_{\sigma_2}\exp{\left( K\sigma_1\sigma_3 + H\sigma_2 + K\sigma_2\sigma_3 \right)}                                                                    \\
   & \hspace{4cm} \times \sum_{\sigma_4}\exp{\left( K\sigma_3\sigma_4 + H\sigma_4 + K\sigma_4\sigma_5 \right)}                                             \\
   & \hspace{4cm} \times \cdots                                                                                                                            \\
   & \hspace{4cm} \times \sum_{\sigma_N}\exp{\left( K\sigma_{N-1}\sigma_N + H\sigma_N + K\sigma_N\sigma_1 \right)}                                         \\
   & = \sum_{\{\sigma_{\text{odd}}\}} \exp{\left( H\sum_{i=1}^{N/2} \sigma_{2i-1} \right)}
  \exp{\left( A' + \frac{\delta H}{2}\sigma_1 + K'\sigma_1\sigma_3 + \frac{\delta H}{2}\sigma_3 \right)}                                                   \\
   & \hspace{4cm} \times \exp{\left( A' + \frac{\delta H}{2}\sigma_3 + K'\sigma_3\sigma_5 + \frac{\delta H}{2}\sigma_5 \right)}                            \\
   & \hspace{4cm} \times \cdots                                                                                                                            \\
   & \hspace{4cm} \times \exp{\left( A' + \frac{\delta H}{2}\sigma_{N-1} + K'\sigma_{N-1}\sigma_1 + \frac{\delta H}{2}\sigma_1 \right)}                    \\
   & = \sum_{\{\sigma_{\text{odd}}\}} \exp{\left( H\sum_{i=1}^{N/2} \sigma_{2i-1} \right)}
  \exp{\left( \frac{NA'}{2} + K'\sum_{i=1}^{N/2}\sigma_{2i-1}\sigma_{2i+1} + \delta H \sum_{i=1}^{N/2}\sigma_{2i-1} \right)}                               \\
   & = \sum_{\{\sigma_{\text{odd}}\}} \exp{\left( \frac{NA'}{2} + K'\sum_{i=1}^{N/2}\sigma_{2i-1}\sigma_{2i+1} + H' \sum_{i=1}^{N/2}\sigma_{2i-1} \right)}
\end{align*}
となる.ここで,$H'=H + \delta H$とした.したがって,1次元イジング模型において部分和をとることで
\begin{equation}
  Z(K,H,N) = e^{N A' / 2} Z(K',H',N/2)
\end{equation}
という関係式が得られる.\par
新しい相互作用$K'$と磁場$H'$の表現を具体的に求める.ここでは,転送行列$\hat{T}$の2乗を計算することで求める.
\begin{align}
  \hat{T}^2
   & =
  \begin{pmatrix}
    e^{K+H} & e^{-K}  \\
    e^{-K}  & e^{K-H} \\
  \end{pmatrix}
  =
  \begin{pmatrix}
    e^{2K+2H} + e^{-2K} & e^{H} + e^{-K}      \\
    e^{H} + e^{-K}      & e^{2K-2H} + e^{-2K} \\
  \end{pmatrix}\notag \\
   & = e^{A'}
  \begin{pmatrix}
    e^{K'+H'} & e^{-K'}   \\
    e^{-K'}   & e^{K'-H'} \\
  \end{pmatrix}
\end{align}
より,
\begin{align*}
  e^{4K'} & = \frac{\cosh(2K+H)\cosh(2K-H)}{\cosh ^2H} \\
  e^{2H'} & = e^{2H} \frac{\cosh(2K+H)}{\cosh(2K-H)}   \\
  e^{4A'} & = 16\cosh(2K+H)\cosh(2K-H)\cosh ^2H
\end{align*}
となる.\par
ここで,いくつか特別な場合に$K'$と$H'$の具体的なふるまいをみていく,ただし,ここでは$A'$は自由エネルギーに寄与する量であるが,考えないことにする.
\begin{itemize}
  \item $H=0$
        \begin{equation*}
          e^{2K'} = \cosh(2K),  \ \ \  H'=0
        \end{equation*}
        $K'$の値は$K=0$のとき$0$,$K$の値が正のときは増加する.磁場は$0$から変化しない.
  \item $K=0$
        \begin{equation*}
          K' = 0,  \ \ \  H' = H
        \end{equation*}
        $K$は相互作用がないことを意味するので,部分和をとっても$K$も$H$も影響しないことを意味する.
  \item $K \rightarrow \infty$
        \begin{equation*}
          e^{4K'} \sim \frac{e^{4K}}{4\cosh^2 H}, \ \ \ H' \sim 2H
        \end{equation*}
        $K'$は無限大のままで,$H'$は増加する.ただし,$H=0$のときは$H'=0$で変わらない.
  \item $H \rightarrow \infty$
        \begin{equation*}
          e^{4K'} \rightarrow 1 , \ \ \ H'=H
        \end{equation*}
        $K'$は急速に$0$に近づき,磁場は変化しない.
\end{itemize}
粗視化を何度も繰り返すと,一つの曲線が形成される.
\begin{equation*}
  e^{2K} \sinh H = \text{const}
\end{equation*}

\end{document}