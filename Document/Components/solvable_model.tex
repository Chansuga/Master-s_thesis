\documentclass[a4paper,11pt]{jsarticle}


% 数式
\usepackage{amsmath,amsfonts}
\usepackage{bm}
% 画像
\usepackage[dvipdfmx]{graphicx}


\begin{document}

\title{可解模型}
\author{須賀勇貴}
\date{最終更新日:\today}
\maketitle
西森先生の本の6章をまとめた

\section{2次元イジング模型の高温展開}
次に,二次元イジング模型を扱う.二次元イジング模型はOnsagerによって厳密解が求められた.Onsagerは,転送行列を対角化することで磁場のないときの自由エネルギーを求め,比熱がある温度で発散することを示した.いくつかの解法が知られているが,ここでは,高温展開を用いた解法を扱う.高温展開は,温度が高いとして自由エネルギーを$\beta J$のべきで展開する方法である.十分高温であれば,その展開を数項で打ち切って自由エネルギーの近似値とする.計算が比較的簡単なため,汎用的な手法として用いられているが,近似の正当性に十分注意する必要がある.この近似のみを用いて相転移を調べることはできない.有限項の和から特異性が生じることはないからである.本節では,無限和を計算することにより厳密解を求め,特異性が生じることを示す.\par
なお,
\subsection{高温展開}
$H=0$の場合に高温展開を行う.二次元イジング模型の分配関数は,
\begin{equation}
  Z = \mathrm{Tr}\prod_{\langle i,j \rangle} \exp{(\beta J S_i S_j)}
\end{equation}
と書ける.積はスピンの最近接対についてとる.スピン変数が$S_i^2=1$であることを用いると,
\begin{align}
  Z
   & = \mathrm{Tr} \prod{\langle i,j \rangle} (\cosh{\beta J} + S_i S_j \tanh{\beta J}) \notag               \\
   & = (\cosh{\beta J})^{N_B}\mathrm{Tr} \prod{\langle i,j \rangle}(1 + S_i S_j \tanh{\beta J}) \label{6.26}
\end{align}
と書ける.$N_B=Nz/2=2N$は最近接対の数を表す.$v=\tanh{\beta J}$は有限温度で$1$より小さい非負の量なので,(\ref{6.26})を$v$について次のように展開してみる.
\begin{equation}
  \prod{\langle i,j \rangle}(1 + v S_i S_j)
  = 1 + v\sum_{\langle i,j \rangle} S_i S_j + v^2 \sum_{\langle i,j \rangle, \langle k,l \rangle} S_i S_j S_k S_l + \cdots
\end{equation}
この展開の項数は膨大であるが,スピン和をとると$\mathrm{Tr}S_i=0$のため多くが$0$になる.有限に残るのは各変数$S_i$が偶数べきに練っている項である.各項は図(\ref{})のようにスピンを結ぶボンドをつなぐ線が閉じている(ループになっている)ときのみである.すべてのスピン変数についての和は$2^N$個あるので$\mathrm{Tr}1=2^N$であり,$l$本のボンドで作ることができるループの数を$g(l)$とすると,
\begin{equation}
  Z = 2^N(\cosh{\beta J})^{N_B} \sum_{l=0}^{\infty} g(l) v_l
\end{equation}
と書ける.ただし,$g(0)=1, \ l>N_B$のとき,$g(l)=0$とする.この$g(l)$を求めて和をとることができれば解が得られる.$v$は高温では小さいから,展開を低次の項で打ち切って近似値を得ることができる.これが高温展開の方法である.\par
一次元と二次元のイジング模型では,この無限和を厳密に計算することができる.一次元の場合,周期的境界条件をとるとループを作ることができるのは$l=N$のときの一つだけである.よって
\begin{equation}
  Z = 2^N (\cosh{\beta J})^N (1 + v^N) \rightarrow (2 \cosh{\beta J})^N
\end{equation}
となり,前節の結果と一致する.二次元の場合,ループの計算は非常に複雑である.

\subsection{二次元イジング模型の解}
$N \rightarrow \infty$のときのエネルギー密度は
\begin{equation}
  -\beta f = \ln{[2 \cosh{\beta J}]} + \frac{1}{2} \int_{0}^{2\pi} \frac{d^2 \bm{k}}{(2 \pi)^2} \ln{\left[ 1 - \frac{t}{2}(\cos{k_1} + \cos{k_2}) \right]}
\end{equation}
と書ける.ここでパラメータ$t$を次のように導入した.
\begin{equation}
  t = \frac{2 \sinh{2\beta J}}{\cosh^2(2\beta J)}
\end{equation}
このパラメータのとりうる値は$0 \leq t \leq 1$である.$T=0$で$0$であり,温度を上げると最大$1$になるまで単調増加し,その点を境に単調減少に転じる.以下で見るように,$t=1$となる温度が相転移店を表す.\par
特異性が生じていることを見るために比熱を計算する.まず内部エネルギーを計算すると,
\begin{equation}
  \epsilon = -\frac{J}{\tanh{(2\beta J)}}\left\{1 - [1 - 2\tanh^2{2 \beta J}] \frac{2}{\pi} K(t) \right\}
\end{equation}
となる.$K(t)$は第一種完全楕円積分を表しており,次のようにして出てくる.
\begin{equation}
  \int_0^{2\pi} \frac{d^2 \bm{k}}{(2\pi)^2}\frac{1}{1 - \frac{t}{2}(\cos{k_1}+\cos{k_2})}
  =\frac{2}{\pi} \int_0^{\pi/2} \frac{\theta}{\sqrt{1 - t^2 \sin^2{\theta}}} = \frac{2}{\pi} K(t)
\end{equation}
内部エネルギーを微分することで比熱が得られる.次の楕円積分の微分を用いる.
\begin{equation}
  \frac{d K(t)}{dt} = \frac{E(t)}{t(1-t^2)} - \frac{K(t)}{t}
\end{equation}
\begin{equation}
  E(t) = \int_0^{2\pi} d\theta \sqrt{1 - t^2 \sin^2{\theta}}
\end{equation}
$E(t)$は第二種完全楕円積分である.こうして,比熱は次のように求められる.
\begin{equation}
  c = \frac{4}{\pi}\left[ \frac{\beta J}{\tanh{(2\beta J)}} \right]^2
  \left( K(t) - E(t) - \frac{1}{\cosh^2{(2\beta J)}} \left\{ \frac{\pi}{2} - [1 - 2\tanh^2{(2\beta J)}] K(t) \right\} \right)
\end{equation}
比熱の特異性は楕円積分の特異性を見ることでわかる.$0 \leq t \leq 1$の範囲において,$K(t)$は$t=1$で発散し,$E(t)$で有限値をとる.よって比熱は$t=1$で発散する.これが二次元イジング模型の相転移である.このときの温度は
\begin{equation}
  \sinh(2\beta J) = 1
\end{equation}
より,
\begin{equation}
  \beta_c J = \frac{1}{2} \ln(1+\sqrt{2}) \approx 0.4407, \ \ \ \frac{T_c}{J} \approx 2.269
\end{equation}
と求められる.なお,内部エネルギーは(\ref{})も楕円積分を含むが,発散は生じない.それは楕円積分にかかる係数$1-2\tanh^2(2\beta J)$が$T=T_c$で$0$となり,発散を打ち消すからである.






\end{document}