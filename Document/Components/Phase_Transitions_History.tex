\documentclass[a4paper,11pt]{jsarticle}


% 数式
\usepackage{amsmath,amsfonts}
\usepackage{bm}
% 画像
\usepackage[dvipdfmx]{graphicx}


\begin{document}

\title{相転移・臨界現象とくりこみ群の概論}
\author{須賀勇貴}
\date{最終更新日:\today}
\maketitle
ここでは,相転移・臨界現象とくりこみ群についての概論と歴史についてまとめている.主に高橋,西森先生の「相転移・臨界現象とくりこみ群」の1章を参考に作成した.\par
\vspace{0.5cm}

ここでh,相転移・臨界現象とは何か?,そして臨界現象の顕著な特徴である普遍性とは何か?,などの問題提起を行う.それによって全体像を概観するとともに,歴史的な経緯を簡単にたどる.

\subsection{相転移と臨界現象}
巨視的な物理系の熱平衡状態は,系を構成する粒子などの要素が膨大な数集まることによって作り出される.そうした状態を表すために,温度や圧力,密度など微視的な系の記述には用いられない量を導入する.そして,熱平衡状態は相(phase)という概念によって特徴づけられる.例えば,水は気体・液体・固体という3種類の形態をとる.この各々の状態が相を表す.つまり,同じ系を考えたとしても,複数の相が実現しうる.どの相となるかは温度などの熱力学変数を指定することによって決まる.複数の相が共存することもある.\par
熱力学変数の値を変えたとき,ある相から異なる相への変化が生じることがある.この変化を理論的に記述することが本書の主題の一つである.例えば水の場合,1気圧の圧力の下にある液体が気体,あるいは気体が液体に変化する温度は摂氏100度である.また,液体と固体の間の変化は摂氏0度で起こる.これが相転移(phase transition)である.理論的には,ある熱平衡状態から異なる熱平衡状態に変化させるとき熱力学関数に何らかの特異性が生じれば,相転移が起こったという.熱力学状態の変化の仕方は一通りではない.図\ref()のように,特異性があるかないかが変化の仕方の経路に依存する場合もある.\par
問題は,考えている系がどのようなときにどのような相転移を示すかを理解することである.統計力学的な観点では,問題は二つに分かれる.系を記述する微視的なハミルトニアンはどのようなものかということと,ハミルトニアンが与えられたときそれが相転移をもたらすかどうかを調べることである.前者の問題は,微視的な機構を現象に応じて洞察する必要があり,系統的な手法はほとんどないだろう.基本的に経験や勘を必要とする問題である.あるいはまた,純粋に理論的な観点から微視的な模型を見いだし,現象を予言するというアプローチもあるだろう.\par
ハミルトニアンが定まれば,統計力学の手法によって熱力学関数を得ることが問題となる.熱力学変数が異なる値をとるとき,質的に異なる熱平衡状態が得られるかどうかは自明なことではない.統計力学はそのような系の多様性を記述できる枠組になっているのだろうか?そして,熱力学変数の特異性はどのようにして生じるのだろうか?\par
そもそも,統計力学の単純な原理から多くの現象が導き出される要因は,多体系を扱っている点にある.ハミルトニアンが単純でも,きわめて多数の構成要素が互いに影響を及ぼすことで,多様な非自明な現象が得られる.このような現象は,協力現象または共同現象とよばれる.相転移はそのような現象の典型的な(そして最難度の)例の一つである.\par
相転移は熱力学系の状態変化であるが,それに伴い臨界点(critical point)において観測される現象を臨界現象(critical phenomena)とよぶ.相転移点では熱力学量に特異性が生じるが,特に臨界点における特異性は以下に述べるように特徴的ない性質を有している.実験的には,これらの特異性を観測することによって相転移の特徴を明らかにする.そのため,臨界現象を理とんてきに記述することは重要な課題である.

\subsection{普遍性とくりこみ群}



\end{document}